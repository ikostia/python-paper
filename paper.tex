\documentclass[a4paper, 12pt]{article}

\usepackage{mathtext}
\usepackage[OT1,T2A]{fontenc}
\usepackage[utf8]{inputenc}
\usepackage[english,ukrainian]{babel}

\begin{document}

\section{Вступ}

\subsection{Про цей документ}

Цей документ є аналітичним звітом, створеним на вимогу викладача на факультеті кібернетики у Київському національному університеті. Інформацію, надану у цьому документі не може бути використано у якості посилання на авторитетне джерело. Усі точки зору, приведені нижче, є виключно точками зору автора і не претендують ні на яку абсолютність (хоча й мають під собою певний досвід програмування). Усі висновки, котрі читач робить із цього документу, він робить на свій страх і ризик. 

\subsection{Мова Python}

Мова Python (читається [пайтон])-- це інтерпретовна, мультипарадигменна мова загального призначення із сильною, динамічною типізацією. Під сильною типізацією мови Python розуміється той факт, що у кожен коткретний момент виконання кожна змінна має чітко визначений тип. Неявні приведення типів допускаються. Під динамічною типізацією мови Python мається на увазі той факт, що типи змінних визначаються у процесі виконання програми (так званому, рантаймі), на противагу моменту компіляції у мов із статичною типізацією. Під мультипарадигменністю мови мається на увазі присутність (і заохочуваність) у програмуванні мовою Python використання та змішування трьох парадигм: об'єктно-орієнтованої, функціональної та структурної.

Попри те, що основною нішею, у котрій використовується Python є веб-програмування, мову активно використовують і у інших галузях програмування: обробка сигналів, математичне моделювання, програмування застосунків робочого столу, серверні скрипти, програмування міжмовних зв'язків тощо.

Історія мови Python налічує вже понад двадцять років. Співробітник голандського інституту Ґвідо ван Россум написав першу версію у 1990 році. Задумувалася мова як розширювана скриптова мова для розподіленої ОС Amoeba, проте досягненнням цієї мети справа не обмежилася. Згодом, викристалізувалася основна ідея, що лежить в основі мови Python: написаний код повинен бути простим, читабельним, непереобтяженим. Звісно, у кількох словах ці рушійні ідеї описати важко, тому нижче цьому буде присвячено цілий розділ.



\end{document}