\documentclass[a4paper, 12pt]{article}

\usepackage{mathtext}
\usepackage[OT1,T2A]{fontenc}
\usepackage[utf8]{inputenc}
\usepackage[english,ukrainian]{babel}

\begin{document}

\section{Вступ}

\subsection{Про цей документ}

Цей документ є аналітичним звітом, створеним на вимогу викладача на факультеті кібернетики у Київському
національному університеті. Інформацію, надану у цьому документі не може бути використано у якості посилання на
авторитетне джерело. Усі точки зору, приведені нижче, є виключно точками зору автора і не претендують ні на яку
абсолютність (хоча й мають під собою певний досвід програмування). Усі висновки, котрі читач робить із цього
документу, він робить на свій страх і ризик.

У документі планується розглянути основні риси стандарту мови, систему версіонування із її двома основними
гілками: 2.x та 3.x, різноманітні реалізації: стандратний CPython, реалізація для JVM Jython, реалізація для 
.NET IronPyton, реалізація Pyton на Python під назвою PyPy, реалізація Pyton із підтримкою мікропотоків Stackless. Крім того, буде розглянуто систему розширень мови Python під назваою Cython.

\subsection{Мова Python}

Мова Python (читається [п\'{а}йтон], у спільноті слов'яномовних програмістів 
часто зустрічається вимова [піт\'{о}н])-- 
це інтерпретовна, мультипарадигменна мова загального призначення із сильною,
динамічною типізацією. Під сильною типізацією мови Python розуміється той факт, що у кожен коткретний момент
виконання кожна змінна має чітко визначений тип. Неявні приведення типів допускаються. Під динамічною типізацією
мови Python мається на увазі той факт, що типи змінних визначаються у процесі виконання програми (так званому,
рантаймі), на противагу моменту компіляції у мов із статичною типізацією. Під мультипарадигменністю мови мається
на увазі присутність (і заохочуваність) у програмуванні мовою Python використання та змішування трьох парадигм:
об'єктно-орієнтованої, функціональної та структурної.

Попри те, що основною нішею, у котрій використовується Python є веб-програмування, мову активно використовують і
у інших галузях програмування: обробка сигналів, математичне моделювання, програмування застосунків робочого
столу, серверні скрипти, програмування міжмовних зв'язків тощо.

Історія мови Python налічує вже понад двадцять років. Співробітник голандського інституту Ґвідо ван Россум
написав першу версію у 1990 році. Задумувалася мова як розширювана скриптова мова для розподіленої ОС Amoeba,
проте досягненнням цієї мети справа не обмежилася. Згодом, викристалізувалася основна ідея, що лежить в основі
мови Python: написаний код повинен бути простим, читабельним, непереобтяженим. Звісно, у кількох словах ці
рушійні ідеї описати важко, тому нижче цьому буде присвячено цілий розділ.

На сьогоднішній день (на момент березня 2012 року) мова Python займає дев'яте місце у рейтингу TIOBE,
спустившись на три позиції, порівняно із березнем 2011 року. Згідно Прозорого індексу популярності мов 
(Transparent Language Popularity Index), Python займає друге місце серед скриптових мов, поступившись 
сумнозвісному PHP. З іншого боку, рейтинги, безумовно, не вказують ані на що, окрім популярності мови,
а популярність не можна вважати достатньо адекватним критерієм якості мови. Саме тому нижче у цьому документі 
проводиться спроба навести більш контруктивну оцінку мови, включаючи її основні концепти, принципи, розвиток 
спільноти, плани подальшого розвитку та потенціал. 

\section{Загальні риси мови Python}

У цій частині приводиться спроба описати загальні концепції, що покладені у основу мови Python та саму реалізацію
цих принципів у вигляді синтаксичних та семантичних особливостей мови.

\subsection{The Zen of Python}

Дзеном пітона називають набір із дев'ятнадцяти правил, сформульваних Тімом Пітерсом (автором знаменитого алгоритму
сортування TimSort) та описуючих принципи, котрих притримуються розробники мови та котрих мають притримуватися
програмісти, що використовують мову Python у своїй діяльності. Це -- неформальні правила, тому, строго кажучи, перевірку 
будь-якого коду на відповідність їм провести неможливо. З іншого боку, досвід програмування на Python дозволяє
формулювати суб'єктивну оцінку коду відповідно до цих правил. Для того, щоб прочитати дзен пітона, достатньо надрукувати
\texttt{import this} у інтеракивному інтерпретаторі або знайти PEP-20 у Інтернеті.

Ось ці правила:
\begin{itemize}
    \item Прекрасно -- краще, ніж огидно.
    \item Явно -- краще, ніж неявно
    \item Просто -- краще, ніж складно.
    \item Складно -- краще, ніж заплутано.
    \item Однорівнево -- краще, ніж вкладено.
    \item Розсіяно -- краще, ніж густо.
    \item Читабельність рахується.
    \item Особливі випадки недостаньо особливі, щоб порушувати правила.
    \item Хоча практичність важливіша за чистоту.
    \item Помилки не повинні залишатися непоміченими.
    \item Окрім тих випадків, коли вони явно приховуються.
    \item Перед лицем двозначності потрібно опиратися спокусі гадати.
    \item Повинен бути один -- і бажано лише один -- очевидний спосіб зробити щось.
    \item Хоча, спершу він може не видаватися таким очевидним, якщо ви не голандець.
    \item Зараз -- краще за ніколи.
    \item Хоча ніколи -- часто краще за \emph{прямо} зараз.
    \item Якщо реалізацію важко пояснити, це погана ідея.
    \item Якщо реалізацію легко пояснити, вона може бути гарною ідеєю.
    \item Простори імен -- неймовірно крута ідея, їх просто необхідно використовувати!
\end{itemize}

\subsection{Принципи, покладені в основу синтаксису}

\subsection{Усе - об'єкт або система типів мови Python}

\subsection{Елементи функціонального програмування}

\subsection{Простори імен}

\subsection{GIL}

\subsection{Відмова від реалізації хвостової рекурсії}


\section{Розвиток мови та спільнота Python}

\subsection{Команда ядра Python}

\subsection{Python Enhancement Proposals (PEPs)}

\subsection{Репозиторії та системи розповсюдження пакетів}

\section{Система версіонування стандартів}

\subsection{Гілка 2.x}

\subsection{Гілка 3.x}

\section{Реалізації Python}

\subsection{CPython}

\subsection{Jython}

\subsection{IronPython}

\subsection{PyPy}

\subsection{Stackless}

\section{Системи розширень та доступу до зовнішніх бібліотек}

\subsection{Cython}

\subsection{ctypes}

\end{document}